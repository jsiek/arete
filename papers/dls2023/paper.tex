%%
%% This is file `sample-acmsmall.tex',
%% generated with the docstrip utility.
%%
%% The original source files were:
%%
%% samples.dtx  (with options: `acmsmall')
%% 
%% IMPORTANT NOTICE:
%% 
%% For the copyright see the source file.
%% 
%% Any modified versions of this file must be renamed
%% with new filenames distinct from sample-acmsmall.tex.
%% 
%% For distribution of the original source see the terms
%% for copying and modification in the file samples.dtx.
%% 
%% This generated file may be distributed as long as the
%% original source files, as listed above, are part of the
%% same distribution. (The sources need not necessarily be
%% in the same archive or directory.)
%%
%%
%% Commands for TeXCount
%TC:macro \cite [option:text,text]
%TC:macro \citep [option:text,text]
%TC:macro \citet [option:text,text]
%TC:envir table 0 1
%TC:envir table* 0 1
%TC:envir tabular [ignore] word
%TC:envir displaymath 0 word
%TC:envir math 0 word
%TC:envir comment 0 0
%%
%%
%% The first command in your LaTeX source must be the \documentclass
%% command.
%%
%% For submission and review of your manuscript please change the
%% command to \documentclass[manuscript, screen, review]{acmart}.
%%
%% When submitting camera ready or to TAPS, please change the command
%% to \documentclass[sigconf]{acmart} or whichever template is required
%% for your publication.
%%
%%
\documentclass[acmsmall]{acmart}
\usepackage{semantic}

%%
%% \BibTeX command to typeset BibTeX logo in the docs
\AtBeginDocument{%
  \providecommand\BibTeX{{%
    Bib\TeX}}}

%% Rights management information.  This information is sent to you
%% when you complete the rights form.  These commands have SAMPLE
%% values in them; it is your responsibility as an author to replace
%% the commands and values with those provided to you when you
%% complete the rights form.
\setcopyright{acmcopyright}
\copyrightyear{2023}
\acmYear{2023}
\acmDOI{XXXXXXX.XXXXXXX}


%%
%% These commands are for a JOURNAL article.
%% \acmJournal{JACM}
%% \acmVolume{37}
%% \acmNumber{4}
%% \acmArticle{111}
%% \acmMonth{8}

%%
%% Submission ID.
%% Use this when submitting an article to a sponsored event. You'll
%% receive a unique submission ID from the organizers
%% of the event, and this ID should be used as the parameter to this command.
%%\acmSubmissionID{123-A56-BU3}

%%
%% For managing citations, it is recommended to use bibliography
%% files in BibTeX format.
%%
%% You can then either use BibTeX with the ACM-Reference-Format style,
%% or BibLaTeX with the acmnumeric or acmauthoryear sytles, that include
%% support for advanced citation of software artefact from the
%% biblatex-software package, also separately available on CTAN.
%%
%% Look at the sample-*-biblatex.tex files for templates showcasing
%% the biblatex styles.
%%

%%
%% The majority of ACM publications use numbered citations and
%% references.  The command \citestyle{authoryear} switches to the
%% "author year" style.
%%
%% If you are preparing content for an event
%% sponsored by ACM SIGGRAPH, you must use the "author year" style of
%% citations and references.
%% Uncommenting
%% the next command will enable that style.
\citestyle{acmauthoryear}

\newcommand{\sem}[1]{\llbracket #1 \rrbracket}
\newcommand{\semV}[1]{\mathcal{V}\sem{#1}}
\newcommand{\semE}[1]{\mathcal{E}\sem{#1}}
\newcommand{\ba}{\begin{array}}
\newcommand{\ea}{\end{array}}
\newenvironment{stack}{\ba{@{}l@{}}}{\ea}
\newenvironment{branch}{\left\{\ba{@{}l@{\qquad}l@{}}}{\ea\right\}}
\newenvironment{syntax}{\[\ba{l@{\;\;}lcl}}{\ea\]}
\newcommand{\dotspace}{.\,}
\newcommand{\key}[1]{\ensuremath{\mathtt{#1}}}
\newcommand{\dyn}{\star}
\newcommand{\Dyn}{\ensuremath{\dyn}}
\newcommand{\Int}{\key{int}}
\newcommand{\Bool}{\key{bool}}
\newcommand{\lam}[1]{\lambda #1 \dotspace}
\newcommand{\app}{\,}
\newcommand{\UnitT}{\mathsf{Unit}}
\newcommand{\RefT}[1]{#1\mathtt{*}}
\newcommand{\alloc}[1]{\mathsf{alloc}\, #1}
\newcommand{\deref}[1]{\mathtt{*}\, #1}
\newcommand{\update}[2]{#1\mathop{:=}#2}
\newcommand{\dealloc}[1]{\mathsf{dealloc}\,#1}
\newcommand{\reduce}{\longrightarrow}
\newcommand{\inj}[2]{#1 \,!\, #2}
\newcommand{\proj}[2]{#1 \,?\, #2}
\newcommand{\pair}[2]{\langle #1 , #2 \rangle}
\newcommand{\fst}[1]{\mathsf{fst}\, #1}
\newcommand{\snd}[1]{\mathsf{snd}\, #1}
\newcommand{\unit}{\mathsf{unit}}
\newcommand{\error}{\mathsf{error}}
\newcommand{\dom}[1]{\mathit{dom}(#1)}
\newcommand{\of}{{:}}
\newcommand{\tu}{{\to}}
\newcommand{\mon}[2]{\mathsf{mon}(#1,#2)}
\newcommand{\addr}[2]{#1 @ #2}
\newcommand{\LET}[3]{\mathsf{let}\,#1 = #2 \,\mathsf{in}\,#3}

%%
%% end of the preamble, start of the body of the document source.
\begin{document}

%%
%% The "title" command has an optional parameter,
%% allowing the author to define a "short title" to be used in page headers.
\title{Fractional Contracts for Dynamic Data-Race Freedom}

%%
%% The "author" command and its associated commands are used to define
%% the authors and their affiliations.
%% Of note is the shared affiliation of the first two authors, and the
%% "authornote" and "authornotemark" commands
%% used to denote shared contribution to the research.
\author{Jeremy G. Siek}
%\authornote{Both authors contributed equally to this research.}
\email{jsiek@indiana.edu}
\orcid{0000-0002-9894-4856}
%% \author{G.K.M. Tobin}
%% \authornotemark[1]
%% \email{webmaster@marysville-ohio.com}
\affiliation{%
  \institution{Indiana University}
  \streetaddress{Luddy Hall, 700 N. Woodlawn Avenue}
  \city{Bloomington}
  \state{IN}
  \country{USA}
  \postcode{47408}
}

%% \author{Lars Th{\o}rv{\"a}ld}
%% \affiliation{%
%%   \institution{The Th{\o}rv{\"a}ld Group}
%%   \streetaddress{1 Th{\o}rv{\"a}ld Circle}
%%   \city{Hekla}
%%   \country{Iceland}}
%% \email{larst@affiliation.org}

%%
%% By default, the full list of authors will be used in the page
%% headers. Often, this list is too long, and will overlap
%% other information printed in the page headers. This command allows
%% the author to define a more concise list
%% of authors' names for this purpose.
%\renewcommand{\shortauthors}{Trovato et al.}

%%
%% The abstract is a short summary of the work to be presented in the
%% article.
\begin{abstract}
  Modern type systems provide modular techniques for the static
  enforcement of data-race freedom but we lack modular techniques for
  dynamic enforcement. However, both static and dynamic enforcement is
  needed in modern languages. For example, even though Rust has a
  powerful borrow checker, it also provides dynamic enforcement
  through the \texttt{Rc}, \texttt{RefCell}, and \texttt{Mutex} types.
  These types rely on reference counting, which is fundamentally
  non-modular. In particular, one cannot formulate software contracts
  that enforce ownership protocols in terms of reference counts.  To
  fill this gap, we present a modular dynamic enforcement mechanism
  for data-race freedom based on the notion of fractional permissions
  and runtime contracts.
\end{abstract}

%%
%% The code below is generated by the tool at http://dl.acm.org/ccs.cfm.
%% Please copy and paste the code instead of the example below.
%%
%\begin{CCSXML}
%\end{CCSXML}

%% \ccsdesc[500]{Computer systems organization~Embedded systems}
%% \ccsdesc[300]{Computer systems organization~Redundancy}
%% \ccsdesc{Computer systems organization~Robotics}
%% \ccsdesc[100]{Networks~Network reliability}

%%
%% Keywords. The author(s) should pick words that accurately describe
%% the work being presented. Separate the keywords with commas.
\keywords{fractional permissions, data races, contracts}

%% \received{20 February 2007}
%% \received[revised]{12 March 2009}
%% \received[accepted]{5 June 2009}

%%
%% This command processes the author and affiliation and title
%% information and builds the first part of the formatted document.
\maketitle

%TODO: Road Map/Outline of what's in the paper

\section{Introduction}

\citet{Boyland:2003aa,Zhao:2007aa,Boyland:2010aa,Boyland:2013aa}

Figure~\ref{fig:lambda-frac}

\begin{figure}
  \raggedright
  \begin{syntax}
    \text{Positive Integers} & n & ::= & 1 \mid 2 \mid \ldots \\
    \text{Fractional Permissions} & f & ::= & 0 \mid 1/n \\
    \text{Contracts} & c & ::= & \\
    \text{Terms} & L,M,N & ::= & x \mid \unit \mid \lam{x} N \mid L \app M
       \mid \pair{L}{M} \mid \fst{M} \mid \snd{M} \\
       &&& \mid \LET{x}{M}{N} \\
       &&& \mid \alloc{M} \mid \deref{M} \mid \update{L}{M} \mid \dealloc{M}\\
       &&& \mid \mon{c}{M} \mid \error \\[2ex]
    %% \text{Runtime Terms} & L,M,N & ::= & \ldots \mid \addr{a}{f} \\
    %% \text{Values} & V,W & ::= & \lam{x} N \mid \unit \mid \addr{a}{f}
  \end{syntax}
  Type System (Linear)
  \begin{gather*}
    \inference{}{\Gamma \vdash x : \Gamma(x)} \quad
    \inference{\Gamma,x:A \vdash N : B}
              {\Gamma \vdash \lam{x}N : A \tu B} \\[2ex]
    \inference{\Gamma \vdash M : A}
              {\Gamma \vdash \alloc{M} : \RefT{A}} \quad
    \inference{\Gamma \vdash M : \RefT{A}}
              {\Gamma \vdash \dealloc{M} : \UnitT} \\[2ex]
    \inference{\Gamma \vdash M : \RefT{A} & \Gamma,x:\RefT{A} \vdash N : B}
              {\Gamma \vdash \LET{x}{M}{N} : B}
  \end{gather*}
  Values
  \begin{syntax}
    \text{Values} & V,W & ::= & \lam{x}N \mid \addr{a}{f}
      \mid \unit
  \end{syntax}
  Reduction
  \begin{align*}
    (\lam{x} N) \app W \mid \mu &\reduce N[x:=W] \mid \mu \\
    \alloc{V} \mid \mu &\reduce \addr{a}{1/1} \mid \mu(a := V) & \text{if } a \notin \dom{\mu} \\
    \dealloc{\addr{a}{1/1}} \mid \mu & \reduce \unit \mid \mu - {a} \\
    \deref{\addr{a}{1/n}} \mid \mu &\reduce \mu(a) \mid \mu \\
    \deref{\addr{a}{0}} \mid \mu &\reduce \error \mid \mu \\
    \update{\addr{a}{1/1}}{V} \mid \mu & \reduce \unit \mid \mu(a:= V)
  \end{align*}

  %% \begin{syntax}
  %%   \text{Values} & V,W & ::= & \langle \lam{x}N , \rho \rangle \mid \addr{a}{f}
  %%     \mid \unit
  %% \end{syntax}
  
  %% Abstract Machine   \fbox{$M , \rho, \kappa \longmapsto M , \rho, \kappa$}
  %% \begin{align*}
  %%   x, \rho, \kappa &\longmapsto \rho(x) \div 2, \rho(x := \rho(x) \div 2), \kappa \\
  %%   L \app M, \rho, \kappa &\longmapsto L, \rho, \Box \app M :: \kappa\\
  %%   \lam{x}N, \rho, \kappa &\longmapsto
  %%       \langle \lam{x}N, \rho\div 2 \rangle, \rho \div 2 , \kappa \\
  %%   V, \rho, \Box \app M :: \kappa & \longmapsto M, \rho, V \app \Box::\kappa\\
  %%   W, \rho, \langle \lam{x}N,\rho' \rangle \app \Box :: \kappa &
  %%     \longmapsto \LET{x}{W}{N}, \rho', \kappa
  %% \end{align*}

Division
\begin{align*}
  \addr{a}{f} \div n &= \addr{a}{f \div n} \\[2ex]
  0 \div n &= 0 \\
  (1/k) \div n &= 1/(kn)
  \end{align*}
  
  \caption{The Fractional Lambda Calculus $\lambda^{1/n}_{\mathsf{Ref}}$}
  \label{fig:lambda-frac}
\end{figure}



\bibliographystyle{ACM-Reference-Format}
\bibliography{all}

\end{document}
\endinput

